\chapter{Software Design}

This section highlights the Software Design by providing an overview of both
the Serial library as well as the MSP-Serial library. Diagrams and design
decisions will be shown and elaborated here.

\section{Serial}

The Serial library offers three major components which are configuring, writing
and reading. Lets go over each one of them and explain in further detail how
that looks like 

\subsection{Configuring a File Descriptor}

In Linux or any Unix-like operating system, file descriptors have to be
configured. The Serial library offers some level of flexibility in terms of
options to configure like blocking (return until all bytes have been send or
return until something has been read) or non-blocking mode, the baudrate of the
file descriptor and more. There is also some device-specific settings like physical I/O
pins that needs to be configured. The Serial library uses the Linux RS-485
kernel module because toggling the RS-485 DE pin in software turned out to be
too slow (See Appendix entry in Subsection \ref{subsec:de-bug}).

\begin{figure}[H]
\centering
\begin{tikzpicture}[node distance=2cm, scale=1, transform shape]

    ¦   \node (decision1) [decision, align=center] {Blocked mode};
    ¦   \node (process1) [process, below left=of decision1, align=center] {blocked FD};
    ¦   \node (process2) [process, below right=of decision1, align=center] {non-blocked FD};

    ¦   \draw [arrow] (decision1) -| node [anchor=south west] {yes} (process1);
    ¦   \draw [arrow] (decision1) -| node [anchor=south east] {no} (process2);

        \node (process3) [process, below of=decision1, xshift=-0.4cm, yshift=-2cm, align=center] {Set baudrate};

        \draw [arrow] (process1) |- node [anchor=south west] {} (process3);
        \draw [arrow] (process2) |- node [anchor=south west] {} (process3);

        \node (stop) [startstop, below of=process3, align=center] {device-specific config};
        
        \draw [arrow] (process3) -- node [anchor=south west] {} (stop);


\end{tikzpicture}

\caption{Flow-chart showing the high-level overview of configuring a file descriptor}
\label{fig:serialsocketconfig}

\end{figure}

\newpage
\subsection{Writing}

Writing is fairly easy as it uses the file descriptor to write to the file descriptor
whatever data and length is given.  This means that it is the responsibility of
the user of the Serial library to make sure that no write-overflow will happen
(length is greater than the amount of bytes to write).

\subsection{Reading}
TODO after bug is fixed.

\newpage
\section{MSP-Serial}

The MSP430FR5969 microcontroller has two eUSCI peripherals, namely eUSCI\_A0
and eUSCI\_A1 which can and will be configured to be in UART mode. Each UART
peripheral will be wired to an RS-485 transceiver and hooked to a RS-485 bus.
COMMs is the only exception compared to the rest of the subsystems as it will
have full-duplex RS-485 communication to the OBC. \newline

\subsection{Configuration and setting up}

The first step is configuring the I/O pin and the eUSCI\_$ \text{A}_x$
peripheral attached to the pin. This is shown in Figure
\ref{fig:mspuartconfig}. As of now, the baudrate is configured to be 115200. 

\begin{figure}[H]
\centering
\begin{tikzpicture}[node distance=2cm, scale=1, transform shape]

    ¦   \node (process1) [process, align=center] {Configure I/O pins};
    ¦   \node (process2) [process, below of=process1, align=center] {Configure UART peripheral};
    ¦   \node (process3) [process, below of=process2, align=center] {Enable Rx/Tx interrupt};
    ¦   \node (process4) [startstop, below of=process3, align=center] {Enable UART};


    ¦   \draw [arrow] (process1) -- node [anchor=south west] {} (process2);
    ¦   \draw [arrow] (process2) -- node [anchor=south west] {} (process3);
    ¦   \draw [arrow] (process3) -- node [anchor=south west] {} (process4);

\end{tikzpicture}

\caption{Flow-chart showing the high-level overview of configuring a UART
    peripheral}
\label{fig:mspuartconfig}

\end{figure}

\subsection{Writing}

Writing is fairly easy. Data is written to the UCAxTXBUF register in
a busy-wait fashion. 

\subsection{Reading}

The biggest design is contained in the reading off Serial. That is because the
MSP430 will receive an arbitrary amount of bytes and we need to buffer them.
The Bus Manager can determine how many bytes will be received but that is part
of the Lunar Zebro Bus Manager protocol. After all, after receiving all bytes
the CRC can be calculated and it can be validated if the packet is valid. But
before all of that has happened in the Bus Manager, the MSP-Serial library
needs to have received all the bytes. \newline

The Interrupt-Service-Routine (ISR) is used to read a single byte from the UART
peripheral and buffered. A state-machine is used as well as a timeout.

%TODO: Add the UML state diagram
