\chapter{Message Format}
\label{ch:message_format}
The message format described in this section is derived from the Bluetooth Low Energy (BLE) protocol but customised to the specific needs for Lunar Zebro and its subsystems.
The level of abstraction allows this message format to be used with all subsystems onboard.
Lets now describe from a top-bottom approach how this message format looks like.
Figure \ref{fig:prop5_1} shows how the message format looks like.
Each message starts with a preset preamble {\scriptsize (PRMBL)} that is equal to $0x3A_{16}$.
The source {\scriptsize (SRC)} field indicates the source address of the message whereas the
destination {\scriptsize (DST)} field indicates the destination address of the message.
The Packet Data Unit (PDU) {\scriptsize (PDU)} field is what we will describe in greater detail in the next paragraph.
The CRC {\scriptsize (CRC)} field holds the CRC calculation over all the previous fields.

\begin{figure}[H]
\vspace{.5cm}
\begin{bytefield}[endianness=little,bitwidth=0.057\textwidth,
bitformatting=\fakeeighteenbits]{18}
\bitheader{0-17}\\
\bitbox{1}{\scriptsize PRMBL} &
\bitbox{1}{\scriptsize SRC} &
\bitbox{1}{\scriptsize DST} &
\bitbox{12}{\scriptsize PDU} &
\bitbox{3}{\scriptsize CRC} &
\end{bytefield}

\caption{Generic message format}
\label{fig:prop5_1}

\end{figure}

The PDU {\scriptsize (PDU)} field itself is split up in a header {\scriptsize (HEADER)} field consisting of two bytes which holds data-link layer information and a payload {\scriptsize (PAYLOAD)} field. This payload field is the data of interest being transferred.
The format of the {\scriptsize (PAYLOAD)} field depends on the message type which is defined in the  {\scriptsize (HEADER)} field.
Figure \ref{fig:prop5_2} shows the message format of the {\scriptsize (PDU)} field.

\begin{figure}[H]
\vspace{.5cm}
\begin{bytefield}[endianness=little,bitwidth=0.057\textwidth,
bitformatting=\fakeeighteenbitss]{18}
\bitheader{0-17}\\
\bitbox{2}{\scriptsize HEADER} &
\bitbox{16}{\scriptsize PAYLOAD} &
\end{bytefield}

\caption{The PDU consists of a header- and a payload part}
\label{fig:prop5_2}

\end{figure}

The {\scriptsize (HEADER)} field, which are the first two bytes of the {\scriptsize (PDU)}, contain information that the data-link layer use. Figure \ref{fig:prop5_3} shows the bitrepresentation of the {\scriptsize (HEADER)} field. Since the {\scriptsize (HEADER)} field is 2 bytes in size means that there are 16 bits in this field.

\begin{figure}[H]
\vspace{.5cm}
\begin{bytefield}[endianness=little, bitwidth=6em]{8}
\bitheader[lsb=0]{0-7}\\
\bitbox{2}{TYPE} &
\bitbox{1}{NESN} &
\bitbox{1}{SN} &
\bitbox{1}{MD} &
\bitbox{3}{RFU} \\[3ex]

\bitheader[lsb=8]{8-15} \\
\bitbox{8}{LENGTH} &

\end{bytefield}

\caption{Header field of the PDU in detail}
\label{fig:prop5_3}

\end{figure}


Table \ref{tab:regdesc1} list the description of each bitfield.

\begin{longtable}{ p{0.1\textwidth} p{0.1\textwidth} p{0.75\textwidth} }

    \textcolor{red}{Field} & \textcolor{red}{Bit(s)} & \textcolor{red}{Decription}\\

    \textbf{TYPE} & [0..1] & Defines the type of the message \newline $00_{2}$: Command \newline $01_{2}$: Retransmission \newline $10_{2}$: Reply \newline $11_{2}$: N(ack)\\
    \textbf{NESN} & 2 & Next Expected Sequence Number  \newline See \enquote{Subsystem Message Format Proposal} on Teams\\
    \textbf{SN} & 3 & Sequence Number  \newline See \enquote{Subsystem Message Format Proposal} on Teams \\
    \textbf{MD} & 4 & More Data \newline $0_{2}$: No more data is coming. Last message of the device \newline $1_{2}$: More data is coming. This is not the last message\\
    \textbf{RFU} & [5..7] & Reserved for Future Use  \\
    \textbf{LENGTH} & [8..15] & Length $l$ of the payload section \newline $0 \leq l \leq 255$ are valid values\\


\caption{Bitfield description of the {\scriptsize HEADER} field}
\label{tab:regdesc1}

\end{longtable}

As mentioned earlier in this Chapter, the {\scriptsize (PAYLOAD)} section depends on the type of the message.
the following few sections will list how the {\scriptsize (PAYLOAD)} field looks like for each type of message.

\section{Command-type PDU}
\label{sec:cmd}

A command is a message type that always goes from the host (OBC, PPU) to the controller (other subsystems).
It tells the subsystem to perform a certain functionality.
A command may require an argument or additional data. This is where the {\scriptsize DATA/PARAM} field comes in.

\begin{figure}[H]
\vspace{.5cm}
\begin{bytefield}[endianness=little,bitwidth=0.057\textwidth,
bitformatting=\fakeeighteenbitss]{18}
\bitheader{0-17}\\
\bitbox{2}{\scriptsize COMMAND} &
\bitbox{16}{\scriptsize DATA/PARAM} &
\end{bytefield}

\caption{The {\scriptsize PAYLOAD} section of the PDU when the message type is a command}
\label{fig:prop5_4}

\end{figure}

Table \ref{tab:regdesc2} lists the descriptions of each bit field regarding the {\scriptsize (PAYLOAD)} field.

\begin{longtable}{ p{0.2\textwidth} p{0.1\textwidth} p{0.75\textwidth} }

    \textcolor{red}{Field} & \textcolor{red}{Byte(s)} & \textcolor{red}{Decription}\\

    \textbf{COMMAND} & [0..1] & Defines which command of the subsystem should be executed. \newline The \enquote{Subsystem Message Format Proposal} document hosts a list of commands for each subsystem. \\
    \textbf{DATA/PARAM} & [2..N] & Additional data or parameters concerning the command \newline $0 \leq l \leq 254$ is a valid range and it depends on each command of the subsystem. \newline See the subsystem specific docuementation to see how each command devides \newline the {\scriptsize (DATA/PARAM)} field \\


\caption{Bitfield description of the {\scriptsize PAYLOAD} field}
\label{tab:regdesc2}

\end{longtable}


\section{Retransmission-type PDU}
\label{sec:retrans}

When a host or controller calculates the CRC over the PDU section and detects a mismatch, it can request a retransmission by sending a retransmission-type message.
The \textbf{NESN} and \textbf{SN} have to reflect a retransmission request as well. More about these bits in \enquote{Subsystem Message Format Proposal} on Teams.
There is no need to supply any data for a retransmission request. Hence why \textbf{LENGTH} should be equal to zero and the {\scriptsize (PAYLOAD)} section is empty.

\newpage
\section{Reply-type PDU}

A reply-type message is send by the controller to the host as a response of a command-type message. \newline
Not all commands will return something to the host.
This means that the reply-type message is optional and depends on the subsystem- and command.
Figure \ref{fig:prop5_5} shows the {\scriptsize (PAYLOAD)} field when the message is a reply-type.
The vast majority of commands return data of a small size. Usually in the order of \texttt{uint8\_t} and \texttt{uint32\_t}.
$N$ can be up to 255 bytes which is enough for almost all commands.

\begin{figure}[H]
\vspace{.5cm}
\begin{bytefield}[endianness=little,bitwidth=0.057\textwidth,
bitformatting=\fakeeighteenbitss]{18}
\bitheader{0-17}\\
\bitbox{18}{\scriptsize DATA} &
\end{bytefield}

\caption{The {\scriptsize PAYLOAD} section of the PDU when the message type is a reply}
\label{fig:prop5_5}

\end{figure}
